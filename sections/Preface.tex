% Preface Section

\section{Preface}

\paragraph{   } This book is intended as an overview of the Wind programming language. It discusses the paradigm of flow-based programming, and the principles of it. The book discusses the advantages and disadvantages of flow-based programming, while then proceeding onto the syntax and usage of the Wind language. The book also, in detail, describes the C implementation of the language. The state management, the instructions, the translation of source code, and execution are all covered. 

\par The Wind language is not a fully fledged, general purpose programming language. It is a language with a fundamental set of computation tools and components. Wind was developed with the following goals in mind. \\

\textbf{\emph{Goals}}
\begin{enumerate}
\item An extremely light-weight language that is highly portable.
\item A programming language which does not use dynamic memory allocation.
\item A fluid, highly dynamically typed runtime.
\item A system that allows efficient transfers of immutable data.
\end{enumerate}

\par The most unique element of Wind is that it is a "bare bones" language. It has no abstract syntax tree, no tokenizer, parser, or standard library. It reads and executes instructions directly from source code, which transition and alter several internal buffers. From a high level overview, these buffers pass around data to one another allowing the state of the data in the program to change.