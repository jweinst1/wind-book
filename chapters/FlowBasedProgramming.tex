% flow-based programming chapter

\chapter{Flow-based Programming}

\paragraph{Flow-based} programming is a paradigm of programming that deals with the flow of data over one or more destinations. These destinations could in theory be any sort of structure. For this book, and the implementation of \emph{Wind}, they are only considered buffers; fixed sized arrays controlled with pointers and indices. 
\par This chapter will focus on an abstract perspective of flow-based programming. It will describe the fundamental concepts, such as nodes, data, and flows. It will also discuss the arrangements of how data flows can be constructed, and give some respective examples.
\par This chapter serves as an introduction to flow-based programming, and a preparation for the actual Wind language. 

\section{Data}

\paragraph{  } In flow-based programming, data is considered any of the values or information that are computed or processed. Numbers, integers, strings, characters, lists, are all examples of data. Nearly all programming languages type the data that they contain and compute. At the raw level, data to a computer is simply sequences of $1$'s and $0$'s.

\begin{lstlisting}[style=cpp]
int a* = new int[1];
\end{lstlisting}