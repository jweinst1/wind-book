
\chapter{Setup, Syntax, and Types}

\paragraph{  } In this chapter, the building and setup of Wind from source code will be discussed. We will also go over the basic syntax and data types present in the wind language. The Wind language also features several configurable options prior to compilation. The chapter will explain how to change and use those options for different scenarios.
\par From a high level perspective, Wind is compiled from a repository of C files and header files. This compilation is done via a \emph{Makefile}, which produces an executable. That executable is runnable via a terminal, and can accept command line arguments. Currently, Wind code can be run interactively, in a \emph{REPL}. Wind can also run code from strings, and with different options, produce debug and performance information.

\section{Download and Building}

\paragraph{  } The Wind source code is stored, managed, and maintained in a git repository on GitHub. If you want to download or visit the repository, you can go to the following link: \\
% link to github repo
\\
\url{https://github.com/jweinst1/Wind}

\paragraph{  } In order to download the repository to your local computer, you need to \emph{clone} it. As long as you are on a UNIX operating system, you can run the following command in your terminal:

\begin{verbatim}
$ git clone https://github.com/jweinst1/Wind.git
\end{verbatim}

\emph{Note:The "\$" as in most conventions is not part of the command, it symbolizes the current user of the terminal.}

\par Next, go through and perform each of the following steps to build Wind:

\begin{verbatim}
$ cd Wind
$ make build
\end{verbatim}

\par These commands will create a \emph{bin} directory which contains an executable named \emph{Wind}. This executable is the program that actually translates and executes Wind source code. The Wind repository does not come with any installers, but you can move the executable to any folder you wish. Later in this chapter, the different command line arguments for the executable will be discussed. However, for now, the "-h" flag will print out the health menu:

% help guide print out
\begin{verbatim}
$ ./bin/Wind -h
The Wind Programming Language Help Guide
          save command: 
               The save command allows you to save the current active data 
               to a file path. This file is written in the Wind binary format.
               Using this allows you to save and load an infinite amount of states
               that Wind can operate and run from.
               __example__: push 3 5 -> map ** 4 | + 4 -> save "nums"
               Saved at: nums
               _______________________ 
               A file named 'nums.bwind' will appear on disk. 
          load command: 
               The load command allows you to load binary wind files into the 
               current active data. 
               This command permits Wind to load data from an infinite 
               number of sources, 
               and expand it's processing capabilities to an immense degree. 
               __example__: push 3 5 -> map ** 4 | + 4 -> save "nums" ->
                           clr -> load "nums" -> out
               [ 85 629 ] 
          reduce command: 
               The reduce command fuses data through a flow of operations. 
               It can be used with operations like +.
               __example__: push 3 3 3 3 3 -> reduce + -> out
               [ 15 ] 
          filter command: 
               The filer command restricts data through a flow of operations. 
               It can be used with operations like < or > or !. 
               __example__: push 5 4 -> map + 3 | * 3 -> out -> filter > 22 -> out 
               [ 24 21 ]
               [ 24 ]
           map command: 
               The map command transforms data through a flow of operations. 
               It can be used with operations like +, - and more. 
               __example__: push 5 5 -> map + 3 | * 3 -> out 
               [ 24 24 ]
           push command: 
               The push command appends data to the end of the active data. 
               __example__: push 5 6 7 -> out 
              [ 5 6 7 ]
           out command: 
               The out command prints the entire active data to stdout. 
           clr command: 
               The clr command resets the active data.
\end{verbatim}

\subsection{Compiling Options}

\paragraph{  } Upon running the build commands for Wind, you might see some messages in your terminal that look like this:

\begin{verbatim}
gcc -c -Wall -Iinclude 
-DWindData_BUF_SIZE=50000 
-DWindData_LOAD_SIZE=10000 
-DWindComp_BUF_SIZE=2000 -c 
-o lib/flow/WindState.o src/flow/WindState.c
\end{verbatim}

\par The above message contains several definitions, which are values passed into macros of the C preprocessor. In the general sense, this allows values to be defined inside source code externally upon compilation, versus manually editing source code. In this case, all of the names appearing after $-D$ are such flags. These flags control the amount of memory used in the buffer nodes of the language internally.
\par As discussed in the beginning of this book, a unique feature of Wind is that it does not use dynamically allocated memory, also called "heap" memory. It relies on static buffers that have memory within the data segment of the executable. This allows Wind to be compiled in a scalable fashion, and it's resource be throttled or increased as needed. Additionally, this feature permits Wind to easily run on embedded systems. Although the specific C-level implementation of Wind has not been discussed yet, the following configuration options can be found in the \codeword{Makefile} of the repo: \\

\textbf{\emph{Wind Compiler Configurations}}
\begin{itemize}
\item \textbf{WindData\_LOAD\_SIZE}: \emph{This option controls the size of load buffer. In Wind, the load buffer accounts for the maximum amount of size of the arguments passed to each command. For now, commands are similar to function calls, such as $f(x, y)$. The size of this buffer is generally small, but it can be customized depending on the size of argument lists needed to be processed.}
\item \textbf{WindData\_BUF\_SIZE}: \emph{This option controls the size of the active and inactive buffers. These buffers hold the data in Wind that can actively be extended, cleared, and manipulated with various operations. The size of this option should always be larger than the load or comp buffer options.}
\item \textbf{WindComp\_BUF\_SIZE}: \emph{This option controls the size of the computation buffer. In the Wind flow, this buffer is used to temporarily store and modify data. Think of it as a larger version of a register. This config option must always be less than the load or active and inactive buffer size options.}
\end{itemize}

\par The settings of these configurations vary depend on the sizes of data you will be handling while running Wind code and whether or not you are building Wind for an embedded system. A general relationship between the different options though can be be described as:

$$
C_o << L_o < B_o
$$

where $C_o$ is the comp size option, $L_o$ is the load size option, and $B_o$ is the active and inactive buffer size option. The size of the comp option should always be \emph{far} smaller than the load option.

\section{Command Line Arguments}

\paragraph{  } The Wind executable that comes from building the language has an array of different options it can be run with for different uses. Command line options in this case are also referred to as "flags". These flags can be used alone or with additional arguments, depending on their purpose. For the following examples, the Wind code used will be: $push\: 5 \longrightarrow out$. Although specific commands in the Wind language have not been explained, we can interpret this code as: \emph{push the integer 5 onto the active buffer and write it to stdout}.

\subsection{Running Strings of Code}

\par In order to run a string of Wind source code, you can choose an option that allows you to specify a string of code as one of the command line arguments. The first of this, is the straight \emph{compile} option, written as "-c":

\begin{verbatim}
$ ./bin/Wind -c "push 5 -> out"
[ 5 ]
\end{verbatim}

Upon executing this command, the list of the active buffer is written to stdout. This is the active buffer inside the Wind runtime. The -c option can be used at any time you need to simply run a string of code. However, let's say if you want some more detailed information about the internal state of Wind, like the command, the state of the translator, and more. There is another option, $-d$:

\begin{verbatim}
$ ./bin/Wind -d "push 5 -> out"
_____Wind___Debug_______
..........State.........
Has Error: false
Mode: Command
Command: null

..........Data.........
Load Buffer: -> [ ]
Active Buffer: -> [ 5 ]
Inactive Buffer: -> [ ]

..........Computation.........
Comp Buffer: -> [ ]
________________________
\end{verbatim}

\par Using the $-d$ option runs the string of source code \emph{and} prints out debug information about the state of the Wind executor and translator. The \emph{state} tab above details the mode of translation, the current command found in the source code, and whether or not an error has been detected. This option is useful for visualizing the current data active in different buffers inside Wind. The computation buffer is separated because it always holds temporary values that are in transit between the flow. The exact structures of these buffers will be discussed in a later chapter.
\par Another option that can be used is the $-t$ option. This option runs the string of source code and also times the run. It prints out the time in seconds it took to fully translate and execute the string of code. This is a great option to check performance of the code your running.

\begin{verbatim}
$ ./bin/Wind -t "push 5 -> out"
[ 5 ]
Time: 0.000081
\end{verbatim}

\subsection{The Wind REPL}

\paragraph{  } Aside from command line arguments, Wind also supports a read, eval and print loop (REPL). This REPL can be activated by executing the Wind executable with no arguments. The REPL prompts the terminal for input, allowing many lines of code to be run continuously. This allows an interactive feel into the flow based nature of the Wind language, and is great for testing out new ideas. Below is an example of the REPL being use:

\begin{verbatim}
$ ./bin/Wind
Wind - Version (0.0.1)
The Wind Programming Language REPL
To exit, simply enter 'exit'. For help, run with '-h' flag.
wind> push 5 6 4 3
wind> out
[ 5 6 4 3 ]
wind> map + 5 | -2 | ** 2 -> out
Error: Cannot map argument of type: 'Number'
wind> map + 5 | - 2 -> out          
[ 8 9 7 6 ]
wind> exit
\end{verbatim}

\par The error, in this example was attempting to map a negative number and not the \emph{minus} symbol. Symbols and mapping will be discussed in a later chapter.

\section{Syntax}

\paragraph{  } The Wind language uses a unique syntax that emphasizes a continuous, stream-like grammar. Wind is designed to be read in a piece by piece, intermittent basis. Such that, Wind programs can be simply described as concatenation of commands, their arguments, and the arrow, $\longrightarrow$ tokens. This permits Wind source code to be easily constructed by other processes and programs. Additionally, having stream-like syntax allows code to be called in smaller increments on environments that have tight memory constraints, like embedded systems.

\subsection{Grammar}

The abstract grammar of Wind can be described as follows:

% Grammar listing and figure
\begin{figure}[h]
\begin{align*}
 code &:= C_1 \rightarrow \dots C_n \\
 command &:= name\:\: a_1 \dots a_n \\
 name &:= [a-zA-Z]+ \\
 argument &:= number\: |\: boolean \:|\: none \:|\: string \\
 boolean &:= True \: | \: False \\
 number &:= [0-9]+
\end{align*}
\caption{The syntactic grammar of the Wind language}
\end{figure}

\par However, the grammar of Wind does not have strict, enclosed rules. For example, even the code: \\

\begin{verbatim}
push  4 5 6 ->
\end{verbatim}

will run without any error, because Wind is designed to be read in a continuous fashion, and be readable as a stream. This means that, like the REPL, the state of code can be preserved between each run.  In fact, running the above string of code in the repl, then just the string "out", will write the previously formed active buffer in the first string to stdout. Wind also supports saving it's current internal state to a binary file and reloading that file at later time. The \emph{save} and \emph{load} commands will be discussed later.

\section{Types}

\paragraph{  } The Wind language uses a simple, efficient type system, that is entirely immutable. The typing system allows types to easily be copied, and written over, such as if transition of one data type to another is needed. In comparison to other programming language terminology, all of Wind's types are \emph{primitive}, there are no constructed or initialized types that can be user defined.
\par The most simplistic data type in Wind is the \emph{None} type. It has no real value, and is meant to symbolize the absence of a value.  However, Wind's types also abide by trueness, where types correspond to either \codeword{False} or \codeword{True} bool types depending on their representation. Since \emph{None} essentially implies an absent or "off" state, in operations that result in a bool output, None will be considered as \codeword{False}:

\begin{verbatim}
$ ./bin/Wind -c "push None -> map ! -> out"
[ False ]
\end{verbatim}

\subsection{Bools}

\paragraph{  } The Bool type in Wind represents the boolean states of true and false. Those values are displayed as \codeword{True} and \codeword{False} respectively. Bool types are used extensively for filtering in Wind. For now though, we can refer to the logical \emph{not} operation performed in previous examples. This is one operator that when used with the \emph{map} command, turns other data types into their respective logical NOT booleans:

\begin{verbatim}
$ ./bin/Wind -c "push None True False 0 3 4 7 3 0 0  -> map ! -> out"
[ False False True False False False False False False False ]
\end{verbatim}

Currently in Wind, \emph{numbers} do not abide by a trueness rule, and thus since a number is not a boolean, it always evaluates to the value False. However, Bools can be evaluated to numbers in the cases of number oriented symbols being used in a command like \codeword{map}, see the example below:

\begin{verbatim}
$ ./bin/Wind -c "push 1 2 -> map + True -> out"
[ 2 3 ]
\end{verbatim}

\subsection{Numbers}

\paragraph{Numbers} in Wind represent numerical values. They can represent both floating pointer numbers and integers. Internally, numbers in Wind are double precision floating point numbers that are displayed as an integer if their value corresponds to one. Numbers are an important type in Wind, used for many commands such as reduction, filtering, mapping and more.
\par Numbers can be pushed onto the active buffer like so:

\begin{verbatim}
$ ./bin/Wind -c "push 1 2 0.0005 -> out"
[ 1 2 0.001 ]
\end{verbatim}

\par Wind only displays up the third decimal place for floating point numbers, but the precision, binary value is stored internally.

\subsection{Strings}

