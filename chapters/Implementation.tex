% Implementation chapter

\chapter{Implementation}
% Begins theoream counter for this chapter.
\newtheorem{theorem}{Theorem}

\paragraph{  } The Wind programming language is run with an executable that is compiled from a C source code project. Wind is written in C to make the language highly performant and compile to a small executable size. One of Wind's main goals is for the language to run on embedded systems. However, another intention of Wind is to create a language which can also serve higher level abstractions (mapping, filtering, reducing), without extensive runtime and low level utilities.
\par Additionally, the implementation of Wind in C also ensures that no dynamic memory allocation is used. Only stack and data-segment (\codeword{static} declared in C/C++) memory are used during computation and runtime. The main buffers of Wind's data flow live in the data segment of the Wind executable. Many embedded systems have issues using heap allocation for memory, thus Wind tackles this issue efficiently.
\par The implementation of Wind is broken up into three main parts: \emph{translation}, \emph{execution} and \emph{computation}. Wind processes source code very differently from other languages as it does not have a traditional pipe line. The language does not use abstract syntax trees, nested-syntax, byte code,  or terminal grammar components. It is read as a series of commands and arguments, as explained in previous chapters. Therefore the phases between source code and resulting computation are quite unique. 
\par Translation is the component in which source code is matched against syntax patterns to transition the internal state of Wind, and loads arguments or data to appropriate buffers. Execution is the phase where the command found from translation is invoked with the current data on the load buffer. Lastly, computation is the component of Wind that manipulates the specific binary data formats to evaluate expressions and binary operations.
\par This chapter will detail the C-level implementation of Wind and the concepts behind the design.

\section{Translation}

\paragraph{  } In most programming languages, the source code is first tokenized, then arranged into an abstract syntax tree. This process is commonly referred to as parsing. Parsing traditionally involves looking through characters in source code by a grammar which is ordered by priority. This is useful for languages with a nested structure. Such as functional languages normally require parsing many different elements, such as function parameters, function bodies, variables, names and more. All of which only occur under certain parent elements.
\par Wind has no nested structure or syntax. It is a truly \emph{linear} language. This means instead of being concerned with the state of the syntactic structure, we are only concerned with the source code itself. For checking a direct \emph{match} for a sequence of characters within an input string, naive comparison or a form of regular expression is used. In Wind however, both of these approaches would be far too slow.
\par Naive matching of a collection of substrings against a target string has a high complexity. Regular expressions can indeed be fast, yet normally require heavy duty utilities. Regular expressions are usually compiled as non-deterministic state machines prior to being able to match or search strings. This is tricky to due in a small scale environment. More so, the syntactic elements of Wind are not overly complex. The true power of regex is revealed in the matching and searching of more intricate structures.
\par Wind uses a technique called \emph{static jump prefixing}, which uses nested jump tables to match prefixes of strings. These jump tables are created when the C-compiler detects a switch statement with 5 or more cases. Most modern C-compilers, including gcc, support this. Nested jump tables allow a syntactic element in source code to be translated to an instruction in $O(N_c)$, where $N_c$ is the number of characters in the element of the source code. The \emph{static} word in the title refers to the jump tables being formed at compile time, meaning they cannot be made or changed during the runtime of the executable. Dynamic jump tables are possible to implement, but would require the tedious task of compiling and running assembly code on the fly. Either way, the algorithm will work the same. To get a better understanding, a jump table is best described as a function, mapping some input to some output to a location within a set.

\begin{theorem}[Jump Tables]
Let $j(x)$ be a piecewise function who's domain is any non-negative integer $x$. Assume that $j$ maps to some position $y$, where $j(x) \longrightarrow y$. Assume that $y$ is either a valid position, or is the null position, $0$.
\end{theorem}

Next, let's say our function $j$ has 2 possible positions, and the null position. The following is a valid piecewise representation:


\[ \begin{cases} 
      0 &\longleftarrow x \neq 45 \land x \neq 64 \\
      1 &\longleftarrow x = 45 \\
      2 &\longleftarrow x = 64 
   \end{cases}
\]

If we assume the input of $j(x)$ are 8-bit signed integers or characters, we can see that such a function only gives a valid output if the character matches it's piece wise mapping. In a computational scenario, the zero case of the piecewise function corresponds to a \codeword{default} or error scenario of a jump table. We can compare the piecewise function with the following switch statement.

\begin{lstlisting}[style=customc]
switch(x) {
      case '$': return 1;
      case '*': return 2;
      default:
            fprintf(stderr, "Syntax Error");
            return; 
}
\end{lstlisting}

\par In the above switch statement, characters are used in place of integer literals to convey the idea of characters from the source code being used as an index to match against. However, so far we have only explored the use of a jump table to match a \emph{length one} element. For multi length elements, such as keyword or command names, we have to use nested jump tables. One can approximate the idea of a nested jump table via nested piecewise functions.
\par The following is an example taken from the Wind source code to match the \emph{clr} command:

\begin{lstlisting}[style=numc]
int WindRun_command(const char** code)
{
        while(**code)
        {
                switch(**code)
                {
                case ' ':
                case '\n':
                case '\t':
                case '\v':
                        *code += 1; //white space
                        break;
                case WindRun_COMMENT_SYM:
                        *code += 1;
                        *code = _move_ptr_end_cmnt(*code);
                        break;
                case 'c':
                        switch((*code)[1])
                        {
                        case 'l':
                                switch((*code)[2])
                                {
                                case 'r':
                                        // exec out
                                        *code += 3;
                                        WindState_set_cmd(WindCommand_clr);
                                        goto TRANS_TO_LOAD;
                                default:
                                        WindState_write_err("Expected command symbol, found 'cl%c'", *code[2]);
                                        return 0;
                                }
                                break;
                        default:
                                WindState_write_err("Expected command symbol, found 'c%c'", *code[1]);
                                return 0;
                        }
                        break;
\end{lstlisting}

\par Starting from the top, let's break down the code. This function accepts a \codeword{const char*} to source code, which it moves as the translation finds more elements, piece by piece. The function's while loop will execute until the source code reaches the null terminating character. When the function encounters certain characters, it may call specialized translation functions to match specific elements, such as an integer or quote-bounded string. 
\par This function ignores whitespace, but considers it a length one syntactic element of any white space character, and advances the source code by one.
\par If a character is encountered the matches 'c', then  the look-ahead character is then checked against a nested switch statement. The main concept here is that, \emph{both} switch statements are jump tables, and nested tables are only reachable through their outer tables. This allows advancement in state towards matching an element in $O(1)$ complexity, one for each character matched. If we want to model this mathematically, let's consider the following:

\begin{theorem}[Jump Prefixing]
Let $j(x) \longrightarrow T$ be a function with a domain of any non-negative integer, $x$. Let $S_t$ be a set of functions. The range of $j(x)$ are the members of set $S_t$. Every function member of $S_t$ subsequently has a range of more sets of functions. This continues until the range of a function results in a termination symbol. 
\end{theorem}

We can further explore this idea in the following diagram:

\begin{tikzcd}[row sep=huge]
& &j_1(x) \arrow[ld, "2"] \arrow[d, "3"] \arrow[rd , "4"]  & & \\
&j_2(x) \arrow[rd, "5"] &j_3(x) \arrow[d, "5"] &j_4(x)\arrow[ld, "5"] \arrow[rd, "7"] & \\
&j_8(x) &j_5(x) &j_6(x) &j_7(x)
\end{tikzcd}

The above is just an example, yet illustrates the nested nature of such switch statements.

\section{Execution and Computation}

\paragraph{  } Once the Wind runtime has translated source code into relevant instructions, Wind proceeds to form it's state information and compute data. For Wind, there are three distinct phases of execution, each which change the context in which incoming code is used. Violations of these execution states will lead to errors being raised. Wind is well suited for running in a REPL-like fashion. It reads code, executes the code, and prints out a response or simply issues a computation to be made.
\par In this section we will discuss the different components that structure Wind's execution model. The computation buffer, and movement of data during computation will also be explained.

\subsection{States}

\paragraph{  } Wind operates in a \emph{mode-like} manner by focusing it's internal state on what next action it should anticipate to do. The State of Wind and the functionality behind switching states is contained within the \codeerror{WindState} module and related header file. Within the c file and it's header, you can find many functions related to changing the state, reading the state, and printing the state (for debugging or string functions). The actual data type that contains Wind's state is an enum:

\begin{lstlisting}[style=customc]
typedef enum
{
        WindMode_command,
        WindMode_load,
        WindMode_exec
} WindMode;
\end{lstlisting} 

The value of the current state is stored as this enum type in a \codeword{static} variable inside WindState.c. That state variable is meant to be accessible and consistent throughout the program, as it governs all other functionality in Wind's runtime. \\

\textbf{\emph{Wind Modes}}
\begin{itemize}
\item \emph{command}: Accepts source code input and scans for a command name. One a command is found, transitions to looking for arguments.
\item \emph{load}: This mode looks and scans for data values and arguments to load onto the load buffer. Once a $->$ arrow symbol is found, the mode transitions to exec.
\item \emph{exec}: This mode uses the computation buffer and related functions to compute the command name and the arguments on the load buffer. This mode also invokes the use of the inactive buffer for the trampoline effect.
\end{itemize}

\par When the \emph{exec} mode is complete, the state simply transitions back to \emph{command}

\subsubsection{Error States}

Wind has a special intermediary state to handle errors. It also uses a \codeword{static} boolean state to indicate if an error has been detected or not. During any of the three main modes,